\documentclass[landscape, 8pt]{extarticle}
\usepackage[a4paper, landscape, hmargin=1cm, tmargin=2cm]{geometry}
\usepackage[utf8]{inputenc}
\usepackage{multicol}
\usepackage{fancyhdr}
\usepackage{amsmath}
\usepackage{environ}
\usepackage{amssymb}
\usepackage{graphicx}
\usepackage[shortlabels]{enumitem}

\newcommand{\Rarr}{\Rightarrow}
\newcommand{\Larr}{\Leftarrow}
\newcommand{\Lrarr}{\Leftrightarrow}
\newcommand{\rarr}{\rightarrow}
\newcommand{\larr}{\leftarrow}
\newcommand{\Var}{\mathrm{Var}}
\newcommand{\E}{\mathbb{E}}
\newcommand{\N}{\mathbb{N}}
\newcommand{\Z}{\mathbb{Z}}
\newcommand{\R}{\mathbb{R}}
\newcommand{\A}{\mathcal{A}}
\newcommand{\Ps}{\mathcal{P}}
\newcommand{\intd}{\text{d}}
\newcommand{\xwarr}{\stackrel{\rarr}{x}}
\newcommand{\Bin}{\mathrm{Bin}}
\newcommand{\Po}{\mathrm{Po}}
\newcommand{\intff}{\int_{-\infty}^\infty}
\newcommand{\Normal}{\mathcal{N}}

\setlength{\columnseprule}{0.2pt}
 
\pagestyle{fancy}
\fancyhf{}
\rhead{Yudhistira Wibowo}
\lhead{DWT Zusammenfassung (Angabe ohne Gewähr!)}
\lfoot{Vergiss nicht, dass die Cheatsheet \textbf{handgeschrieben} werden muss}
\rfoot{Page \thepage}

\NewEnviron{myeq}{%
\begin{equation*}
    \scalebox{0.9}{$\BODY$}
\end{equation*}
}

\begin{document}
\raggedcolumns
\begin{multicols*}{3}


\section{Diskrete Wahrscheinlichkleitraum} 
\subsection{Rechnen mit Ereignissen}
\begin{enumerate}
\item {Ereignissen
\begin{myeq}
A \subseteq B \Rightarrow \Pr[A] \leq \Pr[B]
\end{myeq}
} 
\item{Additionssatz
\begin{myeq}
\forall i \neq j : A_i\cap A_j = \emptyset \Rightarrow \Pr\left[\bigcup_{i=1}^n A_i\right] = \sum_{i=1}^n \Pr[A_i]
\end{myeq}
\textit{"Wahrscheinlichkeit von Vereinigung disjunkter Ereignissen ist nicht anders als die Summe der Wahrscheinlichkeit von den Ereignissen"}
}
\item{Siebformel (Inklusion/Exklusion)
\begin{myeq}
\Pr[A\cup B] = \Pr[A] + \Pr[B]- \Pr[A\cap B]
\end{myeq}
Oder verallgemeint
\begin{myeq}
\Pr\left[\bigcup_{i = 1}^n A_i\right] = \sum_{\emptyset\subset I \subseteq [n]} (-1)^{|I| + 1} \cdot \Pr\left[\bigcap_{i \in I} A_i\right]
\end{myeq}
}
\item{Boolesche Ungleichung
\begin{myeq}
\Pr\left[\bigcup_{i=1}^n A_i\right] \leq \sum_{i=1}^n \Pr[A_i]
\end{myeq}
}
\item {Bedingte Wahrscheinlichkeit
\begin{myeq}
\Pr[Y] > 0 : \Pr[X|Y] = \frac{\Pr[X\cap Y]}{\Pr[Y]} = \frac{|X\cap Y|}{|Y|}
\end{myeq}
}
\item{Satz der totalen Wahrscheinlichkeit
\begin{myeq}
B\subseteq \biguplus_{i=1}^n A_i \Rarr \Pr[B] = \sum_{i=1}^n \Pr[B|A_i] \cdot \Pr[A_i]
\end{myeq}
}
\item{Satz von Bayes
\begin{myeq}
\Pr[B] > 0, B\subseteq \biguplus_{i=1}^n A_i \Rarr \Pr[A_i|B] = \frac{\Pr[B|A_i] \cdot \Pr[A_i]}{\sum_{j=1}^n \Pr[B|A_j] \cdot \Pr[A_j]}
\end{myeq}
}
\item {Multiplikationssatz
\begin{myeq}
\begin{aligned}
\Pr\left[\bigcap_{i=1}^n A_i\right] &= \Pr[A_1] \cdot \prod_{i=2}^n \Pr\left[A_i|\bigcap_{j=1}^{i-1} A_j \right] \\ &= \Pr[A_1] \cdot \Pr[A_2|A_1] \cdot \ldots \cdot \Pr[A_n | A_1 \cap \dots\cap A_{n-1}]
\end{aligned}
\end{myeq}
}
\item{Unabhängigkeit
\begin{myeq}
\Pr[X\cap Y] = \Pr[X] \cdot \Pr[Y]
\end{myeq}
}
\end{enumerate}

\subsection{Zufallsvariable}
\begin{enumerate}
\item {Dichtefunktion
\begin{myeq}
f_X : \R \ni x \mapsto \Pr[X=x] \in [0,1]
\end{myeq}
}
\item {Verteilungsfunktion
\begin{myeq}
F_X:\R\ni x \mapsto \Pr[X \leq x] = \sum_{x'\in W_X: x' \leq x} \Pr[X = x'] \in [0,1]
\end{myeq}
}
\item {Bedingte Zufallsvariable
\begin{myeq}
f_{X|A}(x) = \Pr[X = x~|~A] = \frac{\Pr[X=x \cap A]}{\Pr[A]}
\end{myeq}
}
\end{enumerate}

\subsection{Momenten von Zufallsvariablen}
\begin{enumerate}
\item{Erwartungswert
\begin{myeq}
\begin{aligned}
\E[X] &= \sum_{x\in W_x} x \cdot \Pr[X = x] \\ 
&= \sum_{\omega\in\Omega} X(\omega) \cdot \Pr[\omega] \\
&= \sum_{i = 1}^n \E[X | A_i] \cdot \Pr[A_i] \\
&\stackrel{W_x\subseteq\N_0}{=} \sum_{i=1}^\infty \Pr[X \geq i] \\
\end{aligned}
\end{myeq}
Existienz eines Erwartungswerts: \\ $\sum_{x\in W_x} |x| \cdot \Pr[X = x]$ konvergiert / $\E[X]$ absolut konvergiert.
}
\item {$k$-te Moment
\begin{myeq}
\E[X^k]
\end{myeq}
}
\item{Varianz
\begin{myeq}
\begin{aligned}
\Var[X] &= \E[(X - \E[X])^2] \\
&= \sum_{x\in W_X} \Pr[X = x] \cdot (x - \E[X])^2 \\
&= \E[X^2] - \E[X]^2
\end{aligned}
\end{myeq}
Existienz eines Varianz: \\ $\E[X]$ und $\E[X^2]$ existieren.
}
\item {Standardabweichung
\begin{myeq}
\sigma = \sqrt{\Var[X]}
\end{myeq}
}
\item {$k$-te zentrale Moment
\begin{myeq}
\E[(X - \E[X])^k]
\end{myeq}
}
\end{enumerate}

\subsection{Gesetze zum Rechnen mit Zufallsvariablen}
\begin{enumerate}

\item{Einfache linearität des Erwartungswerts
\begin{myeq}
\E[a\cdot X+ b] = a \cdot \E[X] + b
\end{myeq}
\begin{myeq}
\Var[a \cdot X + b] = a^2 \cdot \Var[X]
\end{myeq}
}
\item{Monotonie des Erwartungswerts
\begin{myeq}
\forall \omega \in \Omega : X(\omega) \leq Y(\omega) \Rarr \E[X] \leq \E[Y]
\end{myeq}
}
\item {Satz 36\\
Seien $X$ eine Zufallsvariable, $A_1, A_2, \dots$ mit $\bigcup_{i=1}^\infty A_i = \Omega$, und $\forall i \in \N : \Pr[A_i] > 0$
\begin{myeq}
\E[X] = \sum_{i=1}^\infty \E[X|A_i] \cdot \Pr[A_i]
\end{myeq}
Sofern die Summe $\sum_{i = 1}^\infty |\E[X|A_i]| \cdot \Pr[A_i]$ konvergiert
}
\item {Mehrere Zufallsvariable\\
Gemeinsame Dichte
\begin{myeq}
f_{X,Y}(x,y) = \Pr[X = x, Y = y]
\end{myeq}
Randdichten
\begin{myeq}
f_X(x) = \sum_{y \in W_Y} f_{X,Y}(x,y)
\end{myeq}
Gemeinsame Verteilung
\begin{myeq}
F_{X,Y}(x,y) = \Pr[X \leq x, Y \leq y] = \sum_{x' \leq x}\sum_{y' \leq y} f_{X,Y}(x', y')
\end{myeq}
Randverteilung
\begin{myeq}
F_X(x) = \sum_{x' \in x} f_{X,Y}(x,y) = \sum_{x' \leq x}\sum_{y \leq W_Y} f_{X,Y}(x', y)
\end{myeq}
Faltungsformel
\begin{myeq}
Z := X + Y \Rarr f_Z(z) = \sum_{x\in W_X} f_X(x) \cdot f_Y(z-x) 
\end{myeq}
}
\item {Unabhängige Zufallsvariable \\ (hier angenommen $X_1,\dots,X_n$ unabhängig)
\begin{myeq}
\Pr[X_1 = a_1, \dots, X_n = a_n] = \Pr[X_1 = a_1] \cdot \ldots \cdot \Pr[X_n = a_n]
\end{myeq}
Multiplikativität des Erwartungswert
\begin{myeq}
\E\left[\prod_{i=1}^n X_i\right] = \prod_{i=1}^n \E[X_i]
\end{myeq}
Varianz einer Summe
\begin{myeq}
\Var\left[\sum_{i=1}^n X_i\right] = \sum_{i = 1}^n \Var[X_i]
\end{myeq}
Satz 46
\begin{myeq}
\forall i \in {1,\dots, n} : S_i\subseteq W_{X_i} \Rarr "X_1 \in S_1",\dots,"X_n\in S_n" \text{ unabhängig}
\end{myeq}
Diese Eigenschaften gelten nur in 1 Richtung! (Also keine gdw. Beziehung!)
}
\item{Linearität des Erwartungswerts
\begin{myeq}
\E\left[\sum_{i=1}^n a_i \cdot X_i\right] =\sum_{i=1}^n a_n\E[X_n] 
\end{myeq}
}
\item {Indikatorsvariable
\begin{myeq}
I_A = \begin{cases}
1 &\text{falls $A$ eintritt} \\
0 &\text{sonst}
\end{cases}
\end{myeq}
Und es gilt 
\begin{myeq}
\E[I_{A_1} \cdot \ldots \cdot I_{A_n}] = \Pr[A_1 \cap \ldots \cap A_n]
\end{myeq}
}
\end{enumerate}

\subsection{Wichtige diskrete Verteilungen}
Sei $p$ wenn Erfolgreich und $q = 1 - p$
\begin{enumerate}
\item {Bernoulli
\begin{myeq}
f_X(x) = 
\begin{cases}
p &x = 1\\
1 - p &x = 0
\end{cases},\quad
\E[X] = p, \quad \Var[X] = pq
\end{myeq}
}
\item{Binomial ($X \sim \Bin(n,p)$)
\begin{myeq}
f_X(x) := b(x;n,p) = \begin{pmatrix} n \\ x \end{pmatrix} \cdot p^x \cdot q^{n-x}, \quad \E[X] = np, \quad \Var[X] = npq
\end{myeq} 
}
\item{Geometrisch
\begin{myeq}
i\in \N : f_X(i) = pq^{i -1}, \quad \E[X] = \frac{1}{p}, \quad \Var[X] = \frac{q}{p^2}
\end{myeq}
Gedächtnislosigkeit
\begin{myeq}
\Pr[X > y + x |X > x] = \Pr[X > y]
\end{myeq}
$n$-ten Erfolg 
\begin{myeq}
f_Z(z) =\begin{pmatrix}z - 1\\n -1\end{pmatrix} \cdot p^n \cdot q^{z-n}
\end{myeq}
($Z$ ist negativ binomialverteilt mit Ordnung $n$) \\ \\
Coupon-Collector-Problem\\
$X_i$ geometrisch verteilt mit:
\begin{myeq}
p = \frac{n - i + 1}{n}, \quad \E[X_i] = \frac{n}{n - i + 1}
\end{myeq}
Es gilt daher:
\begin{myeq}
\E[X] = n \cdot \sum_{i = 1}^n \frac{1}{i}
\end{myeq}
Hypergeometrisch
\begin{myeq}
\Pr[X] = \frac{\begin{pmatrix} b \\ x \end{pmatrix}\begin{pmatrix} a \\ r - x \end{pmatrix}}{\begin{pmatrix} a +b \\ r \end{pmatrix}}
\end{myeq}
}
\item{Poisson ($X \sim \Po(\lambda)$)
\begin{myeq}
i \in \N_0 : f_X(i) = \frac{e^{-\lambda}\lambda^i}{i!}, \quad \E[X] = \lambda, \quad \Var[X] = \lambda
\end{myeq}
Poisson als Grenzwert für Binomial \\ 
($X \sim \Bin(n,\frac{\lambda}{n}) \stackrel{n\rarr\infty}{\longrightarrow} X \sim \Po(\lambda)$)
\begin{myeq}
\lim_{n\rarr \infty} b(k; n, p_n) = \frac{e^{-\lambda} \cdot \lambda^k}{k!}
\end{myeq}
Summe von Poisson Zufallsvariablen
\begin{myeq}
X \sim \Po(\lambda), ~ Y \sim \Po(\mu) \Rarr Z = X + Y \sim \Po(\lambda + \mu)
\end{myeq}
}
\end{enumerate}

\subsection{Abschätzung von Wahrscheinlichkeiten}

\begin{enumerate}
\item{Markov Ungleichung
\begin{myeq}
t > 0 : X \geq 0 \Rarr \Pr[X \geq t] \leq \frac{\E[X]}{t} 
\end{myeq}
}
\item{Chebyshev Ungleichung
\begin{myeq}
t > 0 : \Pr[|X - \E[X]| \geq t] \leq \frac{\Var[X]}{t^2}
\end{myeq}
}
\item{
Gesetzt der großen Zahlen \\
Seien $n$ die Anzahl der Versuch, $X$ eine Zufallsvariable und $X_1, \dots, X_n$ unabhängige Zufallsvariablen mit  derselben Verteilung von $X$,
\begin{myeq}
\epsilon, \delta > 0, ~ \forall n \geq \frac{\Var[X]}{\epsilon\delta^2}, ~ Z = \frac{\sum_{i=1}^n X_i}{n} : \Pr[|Z-\E[X]|\geq \delta] \leq \epsilon 
\end{myeq}
}
\item{Relative Häufigkeit \\
Seien $X$ bernoulliverteilte Indikatorvariable für $A$ mit $\E[X] = p$, $\Pr[A] = p$,
\begin{myeq}
\begin{aligned}
Z &= \frac{|\text{Versuche, bei denen A eingetreten ist}|}{|\text{alle Versuche}|} = \frac{\sum_{i=1}^n X_i}{n} \\ &\Rarr \Pr[|Z-p|\geq \delta] \leq \epsilon
\end{aligned}
\end{myeq}
\begin{myeq}
\text{Relative Abweichung} = \left| \frac{1}{n} \cdot \sum_i X_i - p \right|
\end{myeq}
\begin{myeq}
\text{Absolute Abweichung} = \left|\sum_i X_i - np \right|
\end{myeq}
}
\item{Chernoff-Schranken\\
Es muss für Formel in dieser Nummer gelten: \\
$X_1,\dots,X_n$ unabhängige Bernoulli-verteilte Zufallsvariable mit $\Pr[X_i = 1] = p_i$ und $\Pr[X_i = 0] = 1 - p_i$, $X = \sum_{i=1}^n X_i$, $ \mu = \E[X] = \sum_{i=1}^n p_i$ \\\\
\textit{Upper tail}
\begin{myeq}
\delta > 0 : \Pr[X \geq (1+\delta) \mu] \leq \left(\frac{e^\delta}{(1 + \delta)^{1 + \delta}}\right)^\mu
\end{myeq}
\textit{Lower tail}
\begin{myeq}
0 < \delta < 1 : \Pr[X \leq (1-\delta) \mu] \leq \left(\frac{e^{-\delta}}{(1 - \delta)^{1 - \delta}}\right)^\mu
\end{myeq}
Chernoff-Schranken hängen exponentiell von $\mu$ ab.\\
Lemma 67
\begin{myeq}
0 \leq \delta < 1: (1 - \delta)^{1 - \delta} \geq e^{\frac{\delta^2}{2} - \delta} \text{ und } (1 + \delta)^{1+\delta} \geq e^{\frac{\delta^2}{3} + \delta}
\end{myeq}
Korollar 68
\begin{myeq}
\forall 0 < \delta \leq 1 : \Pr[X \geq (1 + \delta)\mu] \leq e^{-\mu\frac{\delta^2}{3}}
\end{myeq}
\begin{myeq}
\forall 0 < \delta \leq 1 : \Pr[X \leq (1 - \delta)\mu] \leq e^{-\mu\frac{\delta^2}{2}}
\end{myeq}
\begin{myeq}
\forall 0 < \delta \leq 1 : \Pr[|X - \mu| \geq \delta\mu] \leq 2e^{-\mu\frac{\delta^2}{3}}
\end{myeq}
\begin{myeq}
\Pr[X \geq (1 + \delta)\mu] \leq \left(\frac{e}{1 + \delta}\right)^{(1 + \delta)\mu}
\end{myeq}
\begin{myeq}
t \geq 2e\mu : \Pr[X \geq t] \leq 2^{-t}
\end{myeq}
}
\end{enumerate}

\subsection{Erzeugende Funktionen}
\begin{enumerate}
\item {Allgemeine Formel
\begin{myeq}
G_X(s) = \E[s^X] = \sum_{k =0}^\infty \Pr[X = k] \cdot s^k
\end{myeq}
}
\item {Addition mit Konstanten ($Y = X + t$)
\begin{myeq}
G_Y(s) = s^t \cdot \E[s^X] = s^t \cdot G_X(s)
\end{myeq}
}
\item{Aus Summen von Zufallsvariablen \\ 
Feste Summen \\
$X_1, \dots, X_n$ unabhängig und $Z = \sum_{i=1}^n X_i$\\
\begin{myeq}
G_Z(s) = \prod_{i=1}^n G_{X_i}(s)
\end{myeq}
Zufällige Summen \\
$X_1, X_2, \dots$ unabhängig, $N$ unabhängige Zufallsvariable mit wahscheinlichkeiterzeugende Funktion $G_N(s)$ und $Z = \sum_{i=1}^N X_i$\\
\begin{myeq}
G_Z(s) = G_N(G_X(s))
\end{myeq}
}
\item{Ableitungen
\begin{myeq}
\begin{aligned}
G_X'(s) &= \sum_{k=1}^\infty k \cdot \Pr[X = k] \cdot s^{k-1}\\
G_X^{(i)}(0) &= \Pr[X = i] \cdot i!
\end{aligned}
\end{myeq}
}
\item {Erwartungswert
\begin{myeq}
\E[X] = G_X'(1)
\end{myeq}
}
\item {Varianz
\begin{myeq}
\Var[X] = G_X''(1) + G_X'(1) - (G_X'(1))^2
\end{myeq}
}
\item{Satz 71 (Eindeutigkeit) \\Dichte und Verteilung einer Zufallsvariablen $X$ mit $W_X \subseteq \N$ sind durch ihre wahrscheinlichkeiterzeugende Funktion eindeutig bestimmt.\\\\
Gleichverteilt
\begin{myeq}
G_X(s) = \frac{s^{n+1}-1}{(n+1)(s-1)}
\end{myeq}
Bernoulli
\begin{myeq}
G_X(s) = 1 - p + ps
\end{myeq}
Binomial
\begin{myeq}
X \sim \Bin(n,p) \Rarr G_X(s) = (1 - p + ps)^n
\end{myeq}
Geometrisch
\begin{myeq}
G_X(s) = \frac{ps}{1 - (1- p)s}
\end{myeq}
Poisson
\begin{myeq}
X \sim \Po(\lambda) ~ (\text{oder } X \sim \Bin(n,\frac{\lambda}{n})) \Rarr G_X(s) = e^{\lambda(s-1)}
\end{myeq}
}
\end{enumerate}

\subsection{Momenterzeugende Funktion}
\begin{enumerate}
\item {Algemeine Formel
\begin{myeq}
M_X(s) = \E[e^{Xs}] = G_X(e^s)
\end{myeq}
Gleichverteilt auf $[a,b]$
\begin{myeq}
M_U(t) = \frac{e^{tb} - e^{ta}}{t(b -a)} 
\end{myeq}
Normalverteilt $X \sim \Normal(\mu,\sigma^2)$
\begin{myeq}
M_X(t) = e^{\frac{t\mu + (t\sigma)^2}{2}}
\end{myeq}
} 
\item{Aus Summen von Zufallsvariablen \\
($Z = \sum_{i=1}^n X_i$)
\begin{myeq}
M_Z(s) = \prod_{i=1}^n M_{X_i}(s)
\end{myeq}
}
\item{Zufällige Summen\\
($Z = \sum_{i = 1}^N X_i$, $N$ ist Zufallsvariable)
\begin{myeq}
G_Z(s) = G_N(G_X(s))
\end{myeq}
}
\end{enumerate}

\section{Kontinuerliche Wahrscheinlichkeitsräume}
\begin{enumerate}
\item {Allgemein
\begin{myeq}
\int_{-\infty}^{+\infty} f_X(x) \intd x = 1
\end{myeq}
}
\item{Intervalweise
\begin{myeq}
A = \bigcup_k I_k \Rarr \Pr[A] = \int_A f_X(x)\intd x = \sum_k \int_{I_k} f_X(x) \intd x
\end{myeq}
}
\item{Verteilungsfunktion
\begin{myeq}
F_X(x) = \int_{-\infty}^x f_X(t) \intd t
\end{myeq}
}
\item{Eigenschaften
\begin{enumerate}[label= \alph*)]
\item {$F_X$ monoton steigend}
\item {$F_X$ stetig}
\item {$\lim_{x \rarr -\infty} F_X(x) = 0$ und $\lim_{x \rarr \infty} F_X(x) = 1$}
\item {$F$ differenzierbar $\Rarr f(x) = F'(X)$}
\item {$\Pr[a < X \leq B] = F_X(b) - F_X(a)$}
\item {$\int_{[a,b]} f(t) \intd t = \int_{]a,b]} f(t) \intd t = \int_{[a,b[} f(t) \intd t = \int_{]a,b[} f(t) \intd t$}
\end{enumerate}
}
\end{enumerate}

\subsection{$\sigma$-Algebren}
\begin{enumerate}
\item {Definition\\
Sei $\Omega$ Menge, $\A \subseteq \Ps(\Omega)$ heißt $\sigma$-Algebra, wenn alle hier gelten:
\begin{enumerate}[label= E\arabic*)]
\item {
$
\Omega \in \A
$
}
\item {
$
A \in \A \Rarr \overline{A} \in \A
$
}
\item{
$
n \in \N: A_n\in \A \Rarr \bigcup_{n = 1}^\infty A_n \in \A
$
}
\end{enumerate}
Für jede endliche $\Omega$, stellt $\Ps(\Omega)$ eine $\sigma$-Algebra dar.\\
Falls $\Omega = \R$ ist die Klasse der \textit{Borel'schen Mengen}.
}
\end{enumerate}

\subsection{Kolmogorov-Axiome}
\begin{enumerate}
\item{Wahtscheinlichkeitmaß\\
Eine Abbildung $\Pr[.] : \A \rarr [0,1]$ ist Wahrscheinlichkeitmaß auf $\sigma$-Algebren $\A$, wenn sie diese Eigenschaften hat:
\begin{enumerate}[label= W\arabic*)]
\item{
$\Pr[\Omega] = 1$
}
\item {
$A_1,A_2,\dots$ disjunkt $\Rarr \Pr\left[\bigcup_{i=1}^\infty A_i\right] = \sum_{i = 1}^\infty \Pr[A_i]$
}
\end{enumerate}
}
\item {Lemma 84\\
Sei $(\Omega, \A, \Pr)$ Wahrscheinlichkeitraum, für Ereignisse $A, B, A_1, A_2, \dots$ gelten
\begin{enumerate}[label = \alph*)]
\item {
$\Pr[\emptyset] = 0, \Pr[\Omega] = 1$
}
\item{
$0 \leq \Pr[A] \leq 1$
}
\item{
$\Pr[\overline{A}] = 1 - \Pr[A]$
}
\item{
$A \subseteq B \Rarr \Pr[A] \leq \Pr[B]$
}
\item{(Additionssatz) \\
$A_1,A_2,\dots$ disjunkt $\Rarr \Pr\left[\bigcup_{i=1}^\infty A_i\right] = \sum_{i = 1}^\infty \Pr[A_i]$
}
\end{enumerate}
}
\item {Messbarkeit einer Funktion
\begin{myeq}
U : \R, ~B : \R, ~f: U \rarr B \text{ messbar } \Lrarr U, B~ \text{Borel'schen Menge}
\end{myeq}
\begin{enumerate}[label = \alph*)]
\item {Jede stetige Funktion ist messbar}
\item {Jede messbaren Funktion kann man ein Lebesgue Integral ($\int f \intd \lambda$) zuordnen}
\end{enumerate}
}
\end{enumerate}

\subsection{Rechnen mit kontinuerlichen Zufallsvariablen}
\begin{enumerate}
\item {Allgemein
\begin{myeq}
Y = g(X), ~ g : \R \rarr \R, ~C = \{t \in \R, g(t) \leq y\} \Rarr F_Y(y) = \int_C f_X(t) \intd t
\end{myeq}
}
\item {Effekt einer Änderung\\
($Y = a\cdot X + b$)
\begin{myeq}
f_Y(y) = \Pr[aX + b\leq y] = F_X\left(\frac{y -b}{a}\right) = f_X\left(\frac{y - b}{a}\right) \cdot \frac{1}{a}
\end{myeq}
}
\item{Simulation von Zufallsvariablen\\
($\Tilde{X} = F_X^{-1}(U)$)
\begin{myeq}
\Pr[\Tilde{X} \leq t] = F_X(t)
\end{myeq}
}
\item{Kontinuerlich als Grenzwerte von Diskreten
\begin{myeq}
n \in \Z : X_\delta = n \delta \Lrarr X \in [n\delta, (n + 1)\delta]
\end{myeq}
\begin{myeq}
\Pr[X_\delta = n\delta] = F_X((n+1)\delta) - F_X(n\delta)
\end{myeq}
}
\item {Erwartungswert
\begin{myeq}
\int_{-\infty}^\infty |t| \cdot f_X(t) \intd t \text{ endlich } \Rarr \E[X] = \int_{-\infty}^\infty t \cdot f_X(t) \intd t
\end{myeq}
\begin{myeq}
Y = g(X) \Rarr \E[X] = \int_{-\infty}^\infty g(t) \cdot f_X(t) \intd t
\end{myeq}
}
\item {Varianz
\begin{myeq}
\Var[X] = \int_{-\infty}^\infty (t - \E[X])^2 \cdot f_X(t) \intd t \text{ (sofern existiert) }
\end{myeq}
}
\end{enumerate}

\subsection{Wichtige stetige Verteilungen}
\begin{enumerate}
\item {Gleichverteilung
\begin{myeq}
f(x) = \begin{cases}
\frac{1}{b - a} &x \in [a, b]\\
0 &\text{sonst}
\end{cases}, \quad
F(x) = \begin{cases}
0 &x < a\\
\frac{x - a}{b - a} &a \leq x \leq b\\
1 &x > b
\end{cases},
\end{myeq}
\begin{myeq}
\E[X] = \frac{a + b}{2}, \quad
\Var[X] = \frac{(a -b )^2}{12}
\end{myeq}
}
\item {Normalverteilung mit Parameter $\mu \in \R, \sigma \in \R^{+}$ ($X \sim \Normal(\mu, \sigma^2)$)
\begin{myeq}
f(x) = \varphi(x; \mu, \sigma) = \frac{1}{\sqrt{2\pi}\sigma}\cdot \exp{\left(-\frac{(x - \mu)^2}{2\sigma^2}\right)}
\end{myeq}
\begin{myeq}
F(x) = \Phi(x; \mu, \sigma) = \frac{1}{\sqrt{2\pi}\sigma} \cdot \int_{-\infty}^x  \exp{\left(-\frac{(t - \mu)^2}{2\sigma^2}\right)} \intd t
\end{myeq}
\begin{myeq}
I = \int_{-\infty}^\infty e^{-\frac{x^2}{2}} \intd x = \sqrt{2\pi}
\end{myeq}
\begin{myeq}
\E[X] = \mu, \quad \Var[X] = \sigma^2
\end{myeq}
Lineare Transformation
\begin{myeq}
X \sim \Normal(\mu, \sigma^2),~a \in \R^{+},~b\in \R,~Y = aX + b \Rarr Y \sim \Normal(a\mu + b, a^2\sigma^2)
\end{myeq}
Normierte Normalverteilung ($\Normal(0, 1)$)
\begin{myeq}
X \sim \Normal(\mu, \sigma^2), ~Y = \frac{X - \mu}{\sigma} \Rarr Y \sim \Normal(0, 1)
\end{myeq}
\begin{myeq}
X \sim \Normal(\mu, \sigma^2), ~Y = \frac{X - \mu}{\sigma} \Rarr \Pr[a < X \leq b] = \Phi\left(\frac{b - \mu}{\sigma}\right) - \Phi\left(\frac{a - \mu}{\sigma}\right)
\end{myeq}
}
\item {Exponentialverteilung mit Parameter $\lambda > 0$
\begin{myeq}
f(x) = \begin{cases}
\lambda \cdot e^{-\lambda x} &x \geq 0\\
0 &\text{sonst}
\end{cases}, \quad
F(x) = 1 - e^{-\lambda x},
\end{myeq}
\begin{myeq}
\E[X] = \frac{1}{\lambda}, \quad \Var[X] = \frac{1}{\lambda^2}
\end{myeq}
Eigenschaften
\begin{enumerate}[label=\alph*)]
\item {Skalierung \\
$X$ exponentialverteilt mit $\lambda > 0,~Y = aX \Rarr Y$ exponentialverteilt mit $\frac{\lambda}{a}$
}
\item {Gedächtnislosigkeit
\begin{myeq}
\Pr[X > x + y ~|~ X > y] = \Pr[X > x]
\end{myeq}
}
\end{enumerate}
Exponential als Grenzwert Geometrischer\\
Sei $X_n$ geometrisch verteilt mit $p_n = \frac{\lambda}{n}$, dann ist $Y_n = \frac{1}{n}X_n$ exponentailverteilt mit Parameter $\lambda$\\\\
Poisson-Prozess\\
Sei $T_i$ exponentialverteilt mit Parameter $\lambda_T$
\begin{myeq}
X(t) = \max\{n \in \N ~|~ \sum_{i=1}^n T_i \leq t\} \Rarr X(t) \sim \Po(t \cdot \lambda_T)
\end{myeq}
}
\end{enumerate}

\subsection{Mehrere kontinuerliche Zufallsvariable}
\begin{enumerate}
\item {Mehrdimensionale Dichten ($f_{X,Y} : \R^2 \rarr \R^{+}_0$)
\begin{myeq}
f_{X,Y} = \int_{-\infty}^\infty \int_{-\infty}^\infty f_{X,Y}(x,y) \intd x \intd y = 1
\end{myeq}
\begin{myeq}
A \subseteq \R^2 : \Pr[A] = \int_A f_{X,Y}(x,y) \intd x \intd y
\end{myeq}
Bereich
\begin{myeq}
B = \{(x,y) \in \R^2 ~|~ a \leq x \leq b, ~c \leq y \leq d\}
\end{myeq}
}
\item {Gemeinsame Verteilung ($F_{X,Y}:\R^2 \rarr [0,1]$)
\begin{myeq}
F_{X,Y}(x,y) = \Pr[X \leq x, Y \leq y] = \int_{-\infty}^y \int_{-\infty}^x f_{X,Y}(u,v) \intd u \intd v
\end{myeq}
}
\item {Randdichte
\begin{myeq}
f_X(x) = \intff f_{X,Y}(x, v) \intd v
\end{myeq}
}
\item {Randverteilung
\begin{myeq}
F_X(x) = \Pr[X \leq x] = \int_{-\infty}^x \left[\int_{-\infty}^\infty f_{X,Y} (u,v) \intd v  \right] \intd u
\end{myeq}
}
\item {Unabgängigkeit
\begin{myeq}
\forall x, y \in \R : \Pr[X \leq x, ~ Y \leq y] =  \Pr[X \leq x] \cdot \Pr[Y \leq y]
\end{myeq}
\begin{myeq}
F_{X_1, \dots, X_n}(x_1, \dots, x_n) = \prod_{i = 1}^n F_{X_i}(x_i)
\end{myeq}
\begin{myeq}
f_{X_1, \dots, X_n}(x_1, \dots, x_n) = \prod_{i = 1}^n f_{X_i}(x_i)
\end{myeq}
}
\item {Warteproblem\\
Seien $X_1,\dots, X_n$ unabhängig und exponentialverteilt mit Parametern $\lambda_1, \dots, \lambda_n$, dann ist $X = \min\{X_1,\dots,X_n\}$ exponentialverteilt mit $\sum_{i=1}^n \lambda_i$
}
\item {Summen von Zufallsvariablen ($Z = X + Y$)
\begin{myeq}
X, Y \text{ unabhängig} \Rarr f_Z(z) = \intff f_X(x) \cdot f_Y(z - x) \intd x
\end{myeq}
}
\item {Additivität der Normalverteilung \\
Seien $X_1,\dots,X_n$ unabhängig und normalverteilt mit Parametern $\mu_i, \sigma_i$ und $Z = \sum_{i=1}^n a_iX_i$
\begin{myeq}
\E[Z] = \mu = \sum_{i =1}^n a_i\mu_i, \quad \Var[Z] = \sigma^2 = \sum_{i=1}^n a_i^2\sigma_i^2
\end{myeq}
}
\item {Momenterzeugende Funktionen
\begin{myeq}
M_X(s) = \E[e^{Xs}]
\end{myeq}
}
\end{enumerate}

\subsection{Zentraler Grenzwertsatz}

\begin{enumerate}
\item {Allgemein\\
Sei $X_1,\dots, X_n$ haben dieselbe Verteilung und unabhängig mit $\forall i \in \{1,\dots, n\} : \E[X_i] = \mu, \Var[X_i] = \sigma^2$,
\begin{myeq}
n \geq 1 : Y_n = \sum_{i=1}^n X_i \Rarr Z_n = \frac{Y_n - n \mu}{\sigma \sqrt{n}}
\end{myeq}
$Z_n$ ist asymptotisch standardnormalverteilt für $n\rarr \infty$ \\ ($Z_n \sim \Normal(0,1)$)
}
\item {Grenzwertsatz von de Moivre\\ 
Seien $X_1,\dots, X_n$ unabhängige Bernoulliverteilt mit gleicher Erfolgwahrscheinlichkeit $p$ und $H_n = \sum_{i=1}^n X_i$,
\begin{myeq}
\lim_{n\rarr\infty} H_n^* = \lim_{n\rarr\infty} \frac{H_n - np}{\sqrt{np(1-p)}} \sim \Normal(0,1)
\end{myeq}
}
\item {Normalvertilung als Grenzwert Binomialverteilung
\begin{myeq}
H_n\sim \Bin(n,p) : \lim_{n \rarr \infty} \frac{H_n}{n} \sim \Normal\left(p, \frac{p(1-p)}{n}\right)
\end{myeq}
}
\item {Approximationen der Binomialverteilung
\begin{myeq}
\begin{aligned}
H_n \sim \Bin(n, p) : \lim_{n\rarr\infty} F_n(t) &= \lim_{n\rarr\infty} \Pr\left[\frac{H_n}{n} \leq \frac{t}{n}\right] \\ &= \Phi \left(\frac{t-np}{\sqrt{p(1-p)n}}\right) 
\end{aligned}
\end{myeq}
Mit Stetigkeitkorrektur
\begin{myeq}
\lim_{n \rarr \infty} F_n(t) = \Phi \left(\frac{t+ 0,5-np}{\sqrt{p(1-p)n}}\right) 
\end{myeq}
Durch Poisson $\rarr$ gut wenn $np$ sehr klein gegenüber $n$\\
Durch Chernoff-Schranken $\rarr$ sehr gute Ergebnisse\\
Durch Normalverteilung $\rarr$ wenn $np \geq 5$ und $n(1 - p) \geq 5$ (durch $F_n(t)$ Formell oben)
}
\end{enumerate}

\section{Induktive Statistik}
\begin{enumerate}
\item {Schätzer / Schätzvariable\\
Sei $f(x;\theta)$ Dichte von $X$, ein Schätzer für Parameter $\theta$ ist Zufallsvariable, die aus mehreren (meist unabhängig und identisch verteilten) Strichprobenvariablen zusammengesetzt ist.
}
\item {Erwartungstreu
\begin{myeq}
U \text{ erwartungstreu} \Lrarr \E[U] = \theta
\end{myeq}
}
\item {Bias
\begin{myeq}
\text{Bias} = \E[U - \theta]
\end{myeq}
}
\item {Mean Squared Error (MSE)
\begin{myeq}
MSE = \E[(U - \theta)^2]
\end{myeq}
\begin{myeq}
U \text{ erwartungstreu} \Rarr MSE = \Var[U]
\end{myeq}
}
\item {Effizienz
\begin{myeq}
MSE_A < MSE_B \Rarr A \text{ effizienter als } B
\end{myeq}
}
\item {Konsistent im quadratischen Mittel
\begin{myeq}
\lim_{n \rarr \infty} MSE = 0
\end{myeq}
}
\item {Stichprobemittel für Strichprobe $X_1,\dots,X_n$ \\$\rarr$ Erwartungstreu für Erwartungswert
\begin{myeq}
\overline{X} = \frac{1}{n} \sum_{i = 1}^n X_i
\end{myeq}
}
\item {Strichprobevarianz für Strichprobe $X_1,\dots,X_n$ \\$\rarr$ Erwartungstreu für Varianz
\begin{myeq}
S^2 = \frac{1}{n -1} \sum_{i = 1}^n (X_i - \overline{X})^2
\end{myeq}
}
\item {Likelihood-Funktion für Strichprobe $\xwarr$
\begin{myeq}
L(\xwarr; \theta) = \prod_{i = 1}^n f(x_i;\theta)
\end{myeq}
}
\item {Maximum-Likelihood-Schätzwert (MLS) $\hat{\theta}$ für $\xwarr$
\begin{myeq}
\forall \theta: L(\xwarr;\theta) \leq L(\xwarr;\hat{\theta})
\end{myeq}
Tipps: Normalerweise suchen nach Maximal mit Hilfe von $\ln$
}
\item {Konfidenzinterval
\begin{myeq}
\Pr[U_1 \leq \theta \leq U_2] \geq 1 - \alpha
\end{myeq}
$[U_1, U_2]$ ist Konfidenzinterval, symmetrisch wenn $U_1 = U - \delta$ und $U_2 = U + \delta$\\
$1 - \alpha$ ist Konfidenzniveau
}
\item {$\gamma$-Quantil ($x_\gamma$)
\begin{myeq}
X \text{ stetig} : F_X(x_\gamma) = \gamma
\end{myeq}
Für Standardnormalverteilung: $z_\gamma$ und es gilt $z_\alpha = -z_{1 - \alpha}$
}
\end{enumerate}

\subsection{Testen von Hypothesen}
\begin{enumerate}
\item {Kritischen Bereich / Ablehnungsbereich
\begin{myeq}
K = \{\xwarr \in \R^n : \xwarr \text{ führt zur Ablehnung der Hypothese}\}
\end{myeq}
}
\item {Hypothese \\
Zu überprüfende: $H_0$ (Nullhypothese), \\
Alternative: $H_1$
}
\item {Fehler\\
1. Art: $H_0$ gilt, wird aber abgelehnt,\\
Fehlerwahrscheinlichkeit:
\begin{myeq}
\max_{p\in H_0} \Pr[T \in K]$ = $\max_{p\in H_0} \Pr[T \leq K] \approx \Phi\left(\frac{k - np_0}{\sqrt{np_0(1-p_0)}}\right)
\end{myeq}
2. Art: $H_0$ wird nicht abgelehnt, obwohl $H_1$ gilt\\
Fehlerwahrscheinlichkeit:
\begin{myeq}
\sup_{p\in H_1} \Pr[T \notin K]$ = $\sup_{p\in H_1} \Pr[T > K] \approx 1 - \alpha
\end{myeq}
}
\item {Vorgehen bei statistischen Tests
\begin{enumerate}[label= \arabic*. :]
\item Formulierung von Annahmen
\item Formulierung der Nullhypothese
\item Auswahl des Testverfahrens
\item Durchführung des Test und Entscheidung
\end{enumerate}
}
\item {Approximativen Binomialtest \\
Annahme: \\
Seien $X_1, \dots, X_n$ unabhängig bzw. identisch verteilt mit $\Pr[X_i = 1] = p$ und $\Pr[X_i = 0] = 1 - p$ wobei $p$ \textbf{unbekannt} und $n$ hinreichend groß, \\\\
Testgröße: 
\begin{myeq}
Z := \frac{h - np_0}{\sqrt{np_0(1 - p_0)}}
\end{myeq}
Tests:
\begin{center}
\begin{tabular}{|c|c|c|}
    \hline
    $H_0$            & $H_1$         & $K$ \\
    \hline
    $p=p_0$          & $p \neq p_0$  & $|Z| > z_{1-\frac{\alpha}{2}}$ \\
    \hline
    $p\geq p_0$      & $p < p_0$     & $Z < z_\alpha$ \\
    \hline
    $p\leq p_0$      & $p > p_0$     & $Z > z_{1 - \alpha}$ \\
    \hline
\end{tabular}    
\end{center}
}
\item {Gaußtest\\
Annahme: \\
Seien $X_1,\dots,X_n$ unabhängig bzw. identisch verteilt mit $X_i \sim \Normal(\mu, \sigma^2)$ wobei $\sigma^2$ bekannt (und es gelte $\E[X_i] = \mu$ und $\Var[X_i] = \sigma^2$) und $n$ groß genug,\\\\
Testgröße:
\begin{myeq}
Z := \frac{\overline{X} - \mu_0}{\sigma} \sqrt{n}
\end{myeq}
Tests:
\begin{center}
\begin{tabular}{|c|c|c|}
    \hline
    $H_0$            & $H_1$         & $K$ \\
    \hline
    $\mu=\mu_0$          & $\mu \neq \mu_0$  & $|Z| > z_{1-\frac{\alpha}{2}}$ \\
    \hline
    $\mu\geq \mu_0$      & $\mu < \mu_0$     & $Z < z_\alpha$ \\
    \hline
    $\mu\leq \mu_0$      & $\mu > \mu_0$     & $Z > z_{1 - \alpha}$ \\
    \hline
\end{tabular}    
\end{center}
}
\item {$t$-Test\\
Annahme: \\
Seien $X_1,\dots,X_n$ unabhängig bzw. identisch verteilt mit $X_i \sim \Normal(\mu, \sigma^2)$ (und es gelte $\E[X_i] = \mu$ und $\Var[X_i] = \sigma^2$) und $n$ groß genug,\\\\
Testgröße:
\begin{myeq}
T := \frac{\overline{X} - \mu_0}{S} \sqrt{n}
\end{myeq}
Tests:
\begin{center}
\begin{tabular}{|c|c|c|}
    \hline
    $H_0$            & $H_1$         & $K$ \\
    \hline
    $\mu=\mu_0$          & $\mu \neq \mu_0$  & $|T| > t_{n-1,1-\frac{\alpha}{2}}$ \\
    \hline
    $\mu\geq \mu_0$      & $\mu < \mu_0$     & $T < t_{n-1, \alpha}$ \\
    \hline
    $\mu\leq \mu_0$      & $\mu > \mu_0$     & $T > t_{n-1, 1 - \alpha}$ \\
    \hline
\end{tabular}    
\end{center}
Für $n \geq 30$ gelte $t \approx z$
}
\item {2-Strichproben-$t$-Test\\
Annahme: \\
Seien $X_1,\dots,X_m$ und $Y_1, \dots, Y_n$ jeweils unabhängig bzw. identisch verteilt mit $X_i \sim \Normal(\mu_X, \sigma_X^2)$ und $Y_i \sim \Normal(\mu_Y, \sigma_Y^2)$ mit $\Var[X] = \Var[Y]$ oder $\sigma_X^2 = \sigma_Y^2$\\\\
Testgröße:
\begin{myeq}
T := \sqrt{\frac{n + m - 2}{\frac{1}{m} + \frac{1}{n}}} \frac{\overline{X} - \overline{Y}}{\sqrt{(m - 1) \cdot S_X^2 + (n - 1) \cdot S_Y^2}}
\end{myeq}
Tests:
\begin{center}
\begin{tabular}{|c|c|c|}
    \hline
    $H_0$            & $H_1$         & $K$ \\
    \hline
    $\mu=\mu_0$          & $\mu \neq \mu_0$  & $|T| > t_{n + m - 2,1-\frac{\alpha}{2}}$ \\
    \hline
    $\mu\geq \mu_0$      & $\mu < \mu_0$     & $T < t_{n + m - 2, \alpha}$ \\
    \hline
    $\mu\leq \mu_0$      & $\mu > \mu_0$     & $T > t_{n + m - 2, 1 - \alpha}$ \\
    \hline
\end{tabular}    
\end{center}
}
\item {$\chi^2$-Anpassungsest\\
Annahme:\\
Seien $X_1,\dots, X_n$ unabhängig und identisch verteilt mit $W_{X_i} = \{1, \dots, k\}$.\\\\
Hypothese:
\begin{myeq}
H_0: \forall i \in \{1,\dots,k\}:\Pr[X = i] = p_i
\end{myeq}
\begin{myeq}
H_1: \exists i \in \{1,\dots,k\}: \Pr[X = i] \neq p_i 
\end{myeq}
Testgröße:
\begin{myeq}
T = \sum_{i = 1}^k \frac{(h_i + n p_i)^2}{np_i}
\end{myeq}
$h_i$ angibt die Häufigkeit, mit der $X_1, \dots, X_n$ den Wert $i$ annimmt.\\\\
Ablehnungskriterium für $H_0$:
\begin{myeq}
T > \chi_{k-1, 1 - \alpha}^2
\end{myeq}
}
\end{enumerate}

\section{Stochastische Prozesse}
\begin{enumerate}
\item {Stochastisches Prozess = zeitliche Folgen von Zufallsexperimenten $(X_t)_{t\in T}$ \\
$T = \N_0 \rarr$ diskreter Zeit\\
$T = \R_0^{+} \rarr$ kontinuerlicher Zeit
}
\end{enumerate}

\subsection{Markov-Kette}
\begin{enumerate}
\item {Markov-Ketten $\rarr$ nächste Zustand hängt nur vom Aktuellen Zustand ab
}
\item {Zustandmenge 
\begin{myeq}
S = \{0,\dots,n-1\}
\end{myeq}
} 
\item {Startverteilung 
\begin{myeq}
q_0 : q_0^T \in \R^n
\end{myeq}
}
\item {Übergangsmatrix
\begin{myeq}
P = (p_{ij})_{0\leq i, j \leq n}, ~p_{ij} = \Pr[X_{t+1} = j ~|~ X_t = i]
\end{myeq}
}
\item {Wahrscheinlichkeitraum\\
Sei $\omega = (x_0, x_1, \dots, x_{t_0}) \in \Omega$
\begin{myeq}
\Pr[\omega] = (q_0)_{x_0} \cdot \prod_{i=1}^{t_0} \Pr[X_i = x_i~|~ X_{i-1} = x_{i-1}] 
\end{myeq}
}
\item {Berechnung der Übergangswahrscheinlichkeit
\begin{myeq}
(q_{t + 1})_k = \sum_{i = 0}^{n - 1} p_{ik} \cdot (q_t)_i \text{ oder } q_{t + 1} = q_t \cdot P
\end{myeq}
\begin{myeq}
q_{t + k} = q_t \cdot P^k
\end{myeq}
Einfacher: Diagonalisierung!
\begin{myeq}
P^k = B \cdot D^k \cdot B^{-1}
\end{myeq}
}
\item {Übergangszeit / Hitting Time ($i \neq j$)
\begin{myeq}
T_{ij} = \min\{n \geq 0 ~|~ X_n = j \land X_0 = i \},
\end{myeq}
\begin{myeq}
\forall i,j \in S, i \neq j : h_{ij} = \E[T_{ij}] = 1 + \sum_{k\neq j} p_{ik}h_{kj}
\end{myeq}
}
\item {Rückkehrzeit / Recurrence Time ($i = j$)
\begin{myeq}
T_{i} = \min\{n \geq 1 ~|~ X_n = i \land X_0 = i \},
\end{myeq}
\begin{myeq}
h_{i} = \E[T_{i}] = 1 + \sum_{k\neq i} p_{ik}h_{ki}
\end{myeq}
}
\item {Ankunftwahrscheinlichkeit von $i$ nach $j$
\begin{myeq}
f_{ij} = \Pr[T_{ij} < \infty] = p_{ij} + \sum_{k\neq j} p_{ik}f_{kj}
\end{myeq}
}
\item {Rückkehrwahrscheinlichkeit
\begin{myeq}
f_i = \Pr[T_i < \infty] = p_{ii} + \sum_{k\neq i} p_{ik}f_{ki}
\end{myeq}
}
\item {Stationäre Verteilung $\pi$
\begin{myeq}
\pi = \pi \cdot P
\end{myeq}
Tipps: Erstmal prüfen, ob die Markov-Kette ergodisch ist. Wenn ja nimmt die Formel nummer 20.
}
\item {Absorbierende Zustand $i$
\begin{myeq}
\forall j \neq i : p_{ij} = 0, ~p_{ii} = 1 
\end{myeq}
\textit{"$i$ absorbierend, wenn aus $i$ keine Übergänge herausführen"}
}
\item {Transiente Zustand $i$
\begin{myeq}
f_i < 1
\end{myeq}
\textit{"$i$ transient, wenn der Prozess mit positiver Wahrscheinlichkeit $i -f_i > 0$ nach einem Besuch von $i$ nie mehr dorthin zurückkehrt"}
}
\item {Rekurrente Zustand $i$
\begin{myeq}
f_i = 1
\end{myeq}
}
\item {Irreduzibel Markov-Kette
\begin{myeq}
\forall i,j \in S, ~\exists n\in \N: p_{ij}^{(n)} > 0
\end{myeq}
\textit{"Jeder Zustand ist von anderen Zustand aus erreichbar"} oder \textit{"Der Graph ist stark zusammenhängend"}\\
Es folgt auch
\begin{myeq}
\forall f_{ij} = \Pr[T_{ij} < \infty] = 1, \quad h_{ij} = \E[T_{ij}] \text{ existiert}
\end{myeq}
}
\item {Stationäre Verteilung der irreduziblen Markov-Kette
\begin{myeq}
\forall j \in S : \pi_j = \frac{1}{h_{jj}}
\end{myeq}
}
\item {Periode \\ 
Größte Zahl $\xi \in \N$, so dass gilt:
\begin{myeq}
\{n \in \N_0 ~|~ p_{jj}^{(n)} > 0\} \subseteq \{i \cdot \xi ~|~ i \in \N_0\}
\end{myeq}
Falls $\xi = 1$, dann ist der Zustand aperiodisch.\\
Eine Markov-Kette ist aperiodisch, wenn alle Zustände aperiodisch sind.\\
}
\item {Aperiodisch \\
Ein Zustand $i \in S$ ist sicherlich (oder genau dann) aperiodisch, wenn:
\begin{enumerate}[label=\alph*)]
\item eine Schleife besitzt ($p_{ii} > 0$), oder
\item auf mindestens 2 Wegen $W_1$ und $W_2$ liegt, deren Längen $l_1$ und $l_2$ teilfremd sind (ggt($l_1$, $l_2$) = 1). 
\end{enumerate}
Ein Zustand ist gdw. aperiodisch, wenn
\begin{myeq}
\exists n_0 \in \N, ~\forall n \in N, ~n \geq n_0: p_{ii}^{(n)} > 0
\end{myeq}
Aperiodizität einer irreduziblen Markov-Ketten kann durch Schleife für alle Zustände sichergestellt.
}
\item {Korollar 139\\
Für irreduzible, aperiodische Markov-Ketten gilt: 
\begin{myeq}
\exists t \in \N, ~\forall i \in S: (q_t)_i > 0
\end{myeq}
}
\item {Ergodische Markov-Kette = Irreduzible + Aperiodisch
\begin{myeq}
\lim_{n \rarr \infty} q_n = \pi
\end{myeq}
}
\end{enumerate}

\subsection{Doppelstochastische Matrizen}
Sei $P$ eine Matrix der Größe $n\times n$
\begin{enumerate}
\item {Stochastisch
\begin{myeq}
\forall i \in \{0, \dots, n-1\} : \sum_{j=0}^{n-1} p_{ij} = 1
\end{myeq}
\textit{"alle Zeilensumme sind gleich 1."}
}
\item {Doppelstochastisch
\begin{myeq}
P \text{ stochastisch} \land \forall j \in \{0, \dots, n-1\} : \sum_{i=0}^{n-1} p_{ij} = 1
\end{myeq}
\textit{"alle Zeilensumme und Spaltensumme sind gleich 1."}
}
\item {Übergangsmatrix von Markov-Kette ist immer stochastisch
}
\item {Eigenvektor $\pi$ zum Eigenwert 1 bzgl. Multiplikation von links
\begin{myeq}
P \in \R^{n\times n} \text{ doppelstochastisch}, ~\pi = \left(\frac{1}{n}, \dots, \frac{1}{n}\right) : \pi = \pi \cdot P
\end{myeq}
}
\item {Ergodische endliche Markov-Kette $(X_t)_{t \in \N_0} \in \R^{n\times n}$
\begin{myeq}
\forall (X_t)_{t \in \N_0} \text{ doppelstoch. und ergodisch} : \lim_{t \rarr \infty} q_t = \left(\frac{1}{n},\dots,\frac{1}{n}\right)
\end{myeq}
}
\end{enumerate}

\section{Notizen}
\begin{enumerate}
\item {Partitionen von Mengen (Größe $n$) zu disjunkten Teilmengen (Größe $k$)
\begin{myeq}
|\Omega| = \frac{\prod_{i \in \{0\dots\left\lfloor\frac{n}{k}\right\rfloor - 1\}} \begin{pmatrix}n - i \cdot k \\ k\end{pmatrix}}{\left(\left\lfloor\frac{n}{k}\right\rfloor\right)!}
\end{myeq}
}
\item {Geometrische Summe
\begin{myeq}
\sum_{i = 0}^n ar^i = \frac{a\cdot (1 - r)^n}{1 - r}, \quad \sum_{i = 0}^\infty ar^i = \frac{a}{1 - r}
\end{myeq}
}
\item {Konvergierende Summe 
\begin{myeq}
\sum_{i = 1}^\infty \left(\frac{1}{2}\right)^i \approx 1
\end{myeq}
}
\item {Exponentielle Funktion
\begin{myeq}
e^x = \sum_{n = 0}^\infty \frac{x^n}{n!}
\end{myeq}
}
\end{enumerate}


\section{Schluss}
Ich garantiere hiermit die Korrektheit dieser Zusammenfassung nicht. Richtig ist immer die Vorlesung.\\\\
Alle Materialen sind aus der Vorlesung genommen, weswegen es Rechtschutz haben kann!\\\\
Fehler? Verbesserungsvorschlag?\\ 
Kontakt: yudhistira.arief.wibowo@tum.de\\\\
Viel Erfolg bei Klausur! 

\end{multicols*}


\end{document}
